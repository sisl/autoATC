\documentclass[english]{article} 
\setlength  {\textheight}    {9.2in}
\setlength  {\textwidth}     {6.8in}
\setlength  {\oddsidemargin} {0.1in}
\setlength  {\evensidemargin}{0.1in}
\setlength  {\voffset}{-0.6in}
\parindent 0pt 
\parskip 1ex 
\setlength{\unitlength}{1in} 
\usepackage{amsmath, amsfonts, amssymb, amscd} 
\usepackage[usenames,dvipsnames]{color} 
\usepackage{longtable}
\usepackage{verbatim}
\usepackage{listings}
\usepackage{pdfpages}
\usepackage{amsthm,amssymb}
\usepackage[
pdfauthor={Zouhair Mahboubi},
pdfsubject={subject},
pdfpagemode={UseOutlines},
bookmarks,
bookmarksopen,
pdfstartview={FitH},
colorlinks,
linkcolor={blue},
citecolor={blue},
urlcolor={red},
]{hyperref}
\usepackage{fancyhdr} 
\pagestyle{fancy}
\fancyhf{}
\renewcommand{\headrulewidth}{0.5pt}
\fancyfoot[c]{\footnotesize{Project Proposal}}
\fancyfoot[l]{\footnotesize{}}
\fancyfoot[r]{\footnotesize\sffamily\thepage}
\fancypagestyle{plain}{
\renewcommand{\headrulewidth}{0pt}
}
\title{AA228 Project Proposal\\}
\author{Zouhair Mahboubi} 
\begin{document} 
\maketitle 
%\tableofcontents 
% \listoffigures 
\newpage 
\lstset{basicstyle=\footnotesize, breaklines=true, language=Octave}

\section{Motivation}
This project is motivated by a runway incursion recently experienced by the author at an non-towered airport: as I was turning base-to-final to land on runway 30, an aircraft taxied on the opposite end of the runway (12) with the intent to take-off. Luckily he noticed this and initiated a go-around.\\

When recounting this story to fellow pilots, it became apparent that flying at mostly towered airports had provided shelter from the relative anarchy that one sometimes encounters at non-towered airports. Indeed, many pilots are familiar with the dance necessary to enter the pattern at a non-towered airport: the arriving aircraft announces its presence, and relies on other aircraft reporting their positions to try to get a mental picture of which portions of the pattern are occupied in order to infer where the best point of entry is \footnote{The AIM recommends a $45^o$ entry, but other options that might increase separation are often utilized.}\\

This guessing game is removed by the presence of ATC controllers at towered airports, where arriving aircraft are sequenced by directing them to enter at a specified point in the pattern, or aircraft already in the pattern are asked to extend to allow the insertion of a new aircraft. But given the large number of General Aviation airports in the United States, the cost of manning all of them is prohibitive, not to mention that with budget cuts the FAA might even close down already existing ones \footnote{USA Towers closures: http://www.reuters.com/article/2013/03/22/us-usa-towers-closures-idUSBRE92L11J20130322}\\

While a towered airport has its benefits, a lot of pilots enjoy the relative freedom of flying in an uncontrolled airspace. When the pattern is empty, it is indeed liberating; but with other traffic in the air this might lead to mishaps \footnote{Unfortunately there are not many statistics on this as incidents at non-towered airports are not likely to be reported, although according the FAA, General Aviation pilots are often responsible for runway incursions: www.faa.gov/airports/runway safety/ane/faq}. But what if there was a middle-ground? Thanks to FAA efforts in the 80's most non-towered airports were equipped with automated weather stations (AWOS) to improve flight-safety \footnote{AWOS was in response to NTSB reports suggesting that automated weather reports could reduce general aviation accidents. Granted the Controller General did not think that was a good use of taxpayers money: www.gao.gov/products/RCED-85-78}. What if non-towered airports had an 'automated ATC'?\\

We envision the auto-ATC at a non-towered airport to be advisory in nature rather than obligatory (a bit like flight-following for VFR pilots). The pilot would still be responsible for maintaining separation, but the system might be able to provide advise to participating aircraft. One might even imagine such a system being useful at busy general aviation towered airports as an assistive technology to the ATC controller.\\

The topic of autonomous ATC is an active field of research, and a quick literature review turns out recent papers on the topic. A particularly interesting one is 'Autonomous System for Air Traffic Control in Terminal Airspace' by T. Nikoleris, H. Erzberger et Al. It is intended for IFR flights and it utilizes a deterministic approach where candidate 4D trajectories through waypoints are generated and evaluated, and the one with the smallest arrival delay is selected and issued as a command.\\

Here we are proposing a somewhat different approach: a system that issues high-level recommendations (actions) to aircraft tracked through radar (observations) in the immediate vicinity of an airport. The system would aim to maintain separation (rewards) while trying to reduce interventions (cost). The aircraft are expected to be flying according to a model that is influenced by the recommendations, but they are not required to comply with them (i.e. stochastic transitions).

\section{Formulation}
The previous paragraph hints that this might be posed and solved as a POMDP. We informally identify the following States, Actions, Observations and Rewards (the details of conditional probabilities for observations and transitions are TBD) for the process. Note that a lot of simplifying assumptions are made to make this a tractable problem for the purposes of this course:

\begin{itemize}
\item State $S^k$: The state at time-step $t^k$ is an array of N substates {$x_i$}, each associated with an aircraft. The states $x_i$ are themselves broken into:
\begin{enumerate}
\item aircraft location in pattern: $l \in \left\{\mbox{parking, runway, upwind, x-wind, etc.}\right\}$
\item distance left in current location: $d \in \left[0..10\right]$
\item aircraft speed: $s \in \left\{\mbox{slow, fast}\right\}$
\item pilot intentions: $u \in \left\{\mbox{hold, continue, extend, go-around, etc.}\right\}$ 
\end{enumerate}

\item Actions $A^k$: $[a_i^k \in \left\{\mbox{no-op, hold, slowdown, go-around, extend-downwind, etc.}\right\}]$ where $a_i^k$ is the advisory issued at $t^k$ to aircraft $i\in[1..N]$ with the hope of changing its pilot's intention $u_i^{k+1}$.

\item Transitions $T(S^{k+1} | S^{k}, A^{k})$: The transition function would naturally be split into two independent parts, and for simplicty we can start by assuming that aircraft transitions are independent from each other. The first part governs the "physical" states and is independent of the auto-ATC action, and the second one is the advisory's system effect on an the pilot's intentions $u$, i.e.:
$$T(S^{k+1} | S^{k}, A^{k}) = \Pi_i T(S_i^{k+1} | S_i^{k}, a_i^{k})$$
$$T(S_i^{k+1} | S_i^{k}, a_i^{k}) = T(l_i^{k+1}, d_i^{k+1}, s_i^{k+1} | l_i^{k}, d_i^{k}, s_i^{k}, u_i^{k}) T(u_i^{k+1} | {l_i^k, d_i^k, u_i^k} , a_i^k)$$

\item Rewards $R(S^k, A^k)$: The reward function would be designed to increase aircraft separation while minimizing the intervention. We could assume this to be additive, i.e.:
$R(S^k, A^k) = R(S^k) + R(A^k)$ where \footnote{Distance function TBD, A simple distance function would be $0$ if $l_i = l_j$ and $1$ otherwise}
$$ R(S^k) = 
\begin{cases}
-\infty \mbox{ if } min_{i \neq j}(distance(aircraft_i, aircraft_j)) < threshold\\
0 \mbox{    otherwise}\\
\end{cases}
$$

$$R(A^k) = \Sigma R(a_i^k) \mbox{  } \backslash  \mbox{  } R(a_i^k) = 
\begin{cases}
0 \mbox{ if } a_i = \mbox{no-op} \\
-10 \mbox{ if } a_i \in \left\{ \mbox{hold, extend, speed change}\right\} \\
-100 \mbox{ if } a_i \in \left\{\mbox{go-around}\right\}
\end{cases}
$$


\item Observations: \\
($\hat{l}, \hat{d}, \hat{s}$) position and velocity measurements  from a radar system \footnote{This might not require a transponder, one can imagine that a radar might be sufficient, or even a more passive (albeit noisier) option such as RDF en.wikipedia.org/wiki/Radio direction finder}, \\
$\hat{u}_i$ estimate of pilot intent derived from pilot transmissions of intent/position while in pattern\footnote{Digitized through a speech-2-text system}
\end{itemize}

Regarding transitions: since flight near an airport usually follows a somewhat predictable pattern, it makes sense to define the transitions as a Markov Chain. Per example, the chain illustrated in the figure would provide sufficient richness for an interesting problem, while keeping the transition probabilities tractable. Edges labeled in blue are where the auto-ATC has an opportunity to influence the pilot's intent $u_i^{k+1}$. Additionally, auto-ATC can issue slow-down/speed-up commands to change the aircraft speed $s_i^{k+1}$.


\paragraph{\textbf{Solution Approach}} The intended solution approach is to start by trying to solve the problem as an MDP (i.e. assume perfect knowledge of the state) for only 2 aircraft. Depending on problem size, either solve using value iteration or reinforcement learning. If this is successful, attempt handling a larger number of aircraft (5-10 would be reasonable). Time permitting, Relax the full-observability assumption and solve the problem as a POMDP.
\paragraph{\textbf{Evaluation}} there are two evaluation methodologies that might be interesting:
\begin{enumerate}
\item Compare the performance of a candidate policy to a "silent" policy (i.e. no advisories, pilots do as they please) through Monte-Carlo Simulation.
\item Compare the performance relative to a 'human expert', possibly through a simple ATC-like simulation game where a human is allowed to issue the advisories.
\end{enumerate}

\includepdf[pages={1}]{auto-ATC-diagram.pdf}
\end{document}